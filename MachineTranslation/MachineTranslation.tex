\documentclass[12pt,a4paper]{ctexart}

\usepackage[margin=1in, headheight=14pt]{geometry}
\usepackage{amsmath}
\usepackage{amsfonts}
\usepackage{graphicx}
\usepackage{float}
\usepackage{booktabs}
\usepackage{longtable}
\usepackage{hyperref}
\usepackage{fancyhdr}
\usepackage{listings}
\usepackage{xcolor}
\usepackage{tocloft}
\usepackage{tabularx}
\usepackage{multirow}
\usepackage{subcaption}
\usepackage{array}
\usepackage{tcolorbox}
\usepackage{emoji}  % 表情包支持

% --- 智能图片加载命令 ---
\newcommand{\safeincludegraphics}[2][]{%
  \IfFileExists{#2}{%
    \includegraphics[#1]{#2}%
  }{%
    \begin{tcolorbox}[colback=white,colframe=red!50!white,width=\linewidth,halign=center,valign=center,height=5cm]
      \textbf{图片文件未找到}\par
      \texttt{#2}\par
      \small (请确保图片文件位于正确路径)
    \end{tcolorbox}%
  }%
}

% 设置代码样式
\lstset{
    basicstyle=\ttfamily\small,
    breaklines=true,
    captionpos=b,
    numbers=left,
    numberstyle=\tiny\color{gray},
    xleftmargin=1em,
    frame=shadowbox,
    rulesepcolor=\color{red!20!green!20!blue!20},
    commentstyle=\color{green!50!black},
    keywordstyle=\color{blue},
    stringstyle=\color{red!80!black},
    identifierstyle=\color{black},
    backgroundcolor=\color{gray!5},
    title=\lstname,
    language=Python
}

% 图片路径
\graphicspath{{../}{./../}{assets/images/}{./}}

% 超链接配置
\hypersetup{
  colorlinks=true,
  linkcolor=blue,
  urlcolor=blue,
  pdftitle={神经机器翻译实践报告},
  pdfauthor={潘宇轩}
}

% 页眉页脚
\pagestyle{fancy}
\fancyhf{}
\fancyhead[L]{\small \heiti 自然语言处理研讨课(实践课)——第8章 神经机器翻译实践}
\fancyhead[R]{\small \thepage}
\renewcommand{\headrulewidth}{0.4pt}

% 目录美化
\renewcommand{\contentsname}{\heiti\Large 目\hspace{0.5em}录}
\renewcommand{\cftsecfont}{\bfseries\heiti}
\setlength{\cftbeforesecskip}{0.8em}

% 自定义表格列类型
\newcolumntype{Y}{>{\centering\arraybackslash}X}
\newcolumntype{L}{>{\raggedright\arraybackslash}X}

\title{\heiti\textbf{自然语言处理研讨课(实践课)}\\[0.5em]
\Large 第8章 神经机器翻译实践}
\author{\songti\large 潘宇轩\\[0.3em]}
\date{2025年12月9日}

\begin{document}

\maketitle
\thispagestyle{fancy}

% --- 实验成果总结框 ---
\begin{center}
\begin{tcolorbox}[colback=yellow!10,colframe=yellow!50!orange,title=\textbf{实验成果总结},width=0.95\textwidth]
\centering
\begin{tabularx}{\linewidth}{XYcc}
\toprule
\textbf{模型类型} & \textbf{验证集 BLEU} & \textbf{验证集 Loss} & \textbf{备注} \\
\midrule
mBART-50 (预训练) & \textbf{0.51} & 2.90 & 无需微调 \\
Transformer (从零训练) & 0.03 & 7.93 & 10 epochs \\
\bottomrule
\end{tabularx}
\end{tcolorbox}
\end{center}

\vspace{1em}

\tableofcontents
\clearpage

\section{引言:序列建模的挑战}

机器翻译的本质是序列到序列(Sequence-to-Sequence, Seq2Seq)的映射问题。给定一个源语言序列 $X = (x_1, x_2, \dots, x_n)$,我们希望模型能够输出对应的目标语言序列 $Y = (y_1, y_2, \dots, y_m)$。这个问题的难点在于:输入和输出的长度通常不相等,而且两种语言的语法结构、词序可能完全不同。比如英语是 SVO(主谓宾)结构,而日语是 SOV(主宾谓)结构,翻译时需要对词序进行重排。

早期的统计机器翻译(SMT)依赖于大量的人工特征工程和对齐模型,虽然在一定程度上解决了问题,但系统复杂、难以维护。2014 年前后,基于神经网络的端到端翻译模型开始崭露头角,其中最具代表性的就是基于循环神经网络(RNN)的 Encoder-Decoder 架构。然而,随着数据量的爆发和模型深度的增加,RNN 逐渐显露出其时代的局限性。

\section{前世:循环神经网络 (RNN) 的局限}

\subsection{RNN 的基本原理}
RNN 的核心设计理念是"记忆"。它通过一个隐状态(Hidden State)$h_t$ 来承载过去的信息,并按时间步逐步更新。其数学形式通常为:
\begin{equation}
h_t = \tanh(W_{hh} h_{t-1} + W_{xh} x_t + b_h)
\end{equation}
其中 $W_{hh}$ 是隐状态到隐状态的权重矩阵,$W_{xh}$ 是输入到隐状态的权重矩阵。这个公式的含义是:当前时刻的隐状态由上一时刻的隐状态和当前输入共同决定。

在 Encoder-Decoder 架构中,编码器 RNN 将整个源句子压缩成一个固定长度的向量(通常是最后一个隐状态 $h_n$),然后解码器 RNN 以这个向量为初始状态,逐词生成目标句子。这种设计简洁优雅,但也埋下了隐患。

\subsection{时间步的枷锁}
这种设计虽然符合人类阅读文本的直觉(从左到右),但在工程和数学上带来了两个致命问题。

首先是\textbf{无法并行计算}。由于 $h_t$ 的计算依赖于 $h_{t-1}$,这意味着无论 GPU 有多少核心,处理一个长度为 100 的句子必须串行执行 100 次运算。在深度学习时代,数据量动辄上亿,这种串行瓶颈严重限制了训练效率。相比之下,卷积神经网络(CNN)可以在一次矩阵运算中处理整个序列,速度快了不止一个数量级。

其次是\textbf{信息遗忘}(长距离依赖问题)。想象一下,如果源句子是"The cat, which was sitting on the mat and looking at the bird outside the window, suddenly jumped up.",当模型处理到"jumped"时,它需要记住主语是"cat"而不是"bird"或"window"。但经过这么多时间步的传递,关于"cat"的信息可能已经被稀释甚至遗忘了。

LSTM(Long Short-Term Memory)通过引入门控机制(遗忘门、输入门、输出门)来缓解这个问题,但并没有从根本上解决。实验表明,即使是 LSTM,在处理超过 50 个词的句子时,性能也会明显下降。

\subsection{梯度消失与梯度爆炸}
从数学角度看,RNN 的梯度需要通过时间反向传播(Backpropagation Through Time, BPTT)。假设我们要计算 $\frac{\partial L}{\partial h_1}$,需要将梯度从 $h_T$ 一路传回 $h_1$:
\begin{equation}
\frac{\partial L}{\partial h_1} = \frac{\partial L}{\partial h_T} \prod_{t=2}^{T} \frac{\partial h_t}{\partial h_{t-1}}
\end{equation}
如果 $\frac{\partial h_t}{\partial h_{t-1}}$ 的范数小于 1,连乘后梯度会指数级衰减(梯度消失);如果大于 1,则会指数级增长(梯度爆炸)。这使得 RNN 很难学习到长距离的依赖关系。

\subsection{RNN 结构可视化}
为了直观理解这一过程,我们绘制了 RNN 的时间展开图(图 \ref{fig:rnn})。

\begin{figure}[htbp]
  \centering
  \framebox{\parbox{0.9\textwidth}{\centering
    \vspace{0.5cm}
    \includegraphics[width=0.9\textwidth]{images/rnn_high_res.png}
    \vspace{0.5cm}
  }}
  \caption{\textbf{RNN 的时间展开结构。} 左侧为折叠形式,右侧为按时间步展开。可以看到数据必须沿着箭头 $h_{t-1} \to h_t \to h_{t+1}$ 逐步流动,无法并行。}
  \label{fig:rnn}
\end{figure}

如图 \ref{fig:rnn} 所示,信息的流动被严格限制在时间轴上,这是一条单行道。

\section{今生:Transformer 的革命}

2017 年,Google 团队在论文《Attention Is All You Need》中提出了 Transformer 架构。这篇论文的标题本身就是一个宣言:我们不需要循环,不需要卷积,\textbf{注意力机制就是一切}。

\subsection{注意力机制的前身}
其实注意力机制并不是 Transformer 的发明。早在 2014 年,Bahdanau 等人就提出了在 RNN Encoder-Decoder 中加入注意力机制的想法。当时的动机很简单:既然把整个句子压缩成一个固定长度的向量会丢失信息,那为什么不让解码器在生成每个词时,都能"回头看"编码器的所有隐状态呢?

这个想法效果显著,但仍然受限于 RNN 的串行计算。Transformer 的贡献在于:既然注意力机制这么好用,那我们干脆把 RNN 全部换成注意力机制,实现真正的并行计算。

\subsection{核心理念:全局视野与并行计算}
Transformer 的核心思想可以用一句话概括:与其按部就班地传递信息,不如让序列中的每一个词都能直接"看"到其他所有词。

这种机制带来了两大突破。第一是\textbf{并行计算}:由于每个位置的计算不依赖于其他位置的结果,所有词的特征提取可以同时进行。在 GPU 上,这意味着一个长度为 100 的句子可以在一次矩阵运算中完成,而不是串行执行 100 次。第二是\textbf{恒定路径长度}:在 RNN 中,位置 1 的信息要传递到位置 100,需要经过 99 个时间步;而在 Transformer 中,任意两个位置之间的交互距离都是 1,这彻底解决了长距离依赖问题。

\subsection{自注意力机制 (Self-Attention) 详解}
自注意力机制是 Transformer 的灵魂。为了解释它,我们可以使用"搜索引擎"的类比。

想象你在使用搜索引擎:你输入一个查询(Query),搜索引擎会将你的查询与数据库中所有文档的关键词(Key)进行匹配,然后返回最相关的内容(Value)。自注意力机制的工作方式与此类似:对于输入序列中的每一个词,我们将其映射为三个向量——\textbf{Query ($Q$)} 表示"我想找什么信息",\textbf{Key ($K$)} 表示"我包含了什么特征"用于被 $Q$ 匹配,\textbf{Value ($V$)} 表示"我的具体内容是什么"。

具体来说,这三个向量通过三个不同的线性变换得到:
\begin{equation}
Q = XW^Q, \quad K = XW^K, \quad V = XW^V
\end{equation}
其中 $X$ 是输入的词嵌入矩阵,$W^Q, W^K, W^V$ 是可学习的参数矩阵。

注意力权重的计算过程分为三步:首先,计算 $Q$ 和 $K$ 的点积得到相似度分数;然后,除以 $\sqrt{d_k}$ 进行缩放(防止点积值过大导致 Softmax 梯度消失);最后,通过 Softmax 归一化得到注意力权重,并将其作用在 $V$ 上进行加权求和。

\begin{equation}
    \text{Attention}(Q, K, V) = \text{softmax}\left(\frac{QK^T}{\sqrt{d_k}}\right)V
\end{equation}

这个公式看起来简单,但蕴含着深刻的思想:每个词的输出表示是所有词的 Value 的加权和,权重由该词与其他词的相关性决定。这使得模型能够动态地关注句子中最相关的部分。

\begin{figure}[htbp]
  \centering
  \framebox{\parbox{0.9\textwidth}{\centering
    \vspace{0.5cm}
    \includegraphics[width=0.9\textwidth]{images/attn_high_res.png}
    \vspace{0.5cm}
  }}
  \caption{\textbf{自注意力机制可视化。} 单词 "it" (Query) 根据语义相关性,分配了不同的权重给 "The" 和 "animal" (Keys)。线条越粗代表关注度越高,从而解决了指代消解等复杂语义问题。}
  \label{fig:self_attn}
\end{figure}

图 \ref{fig:self_attn} 清晰地展示了这一点:模型在处理 "it" 这个词时,不仅看到了它本身,还通过粗线条"注意"到了 "animal",从而理解 "it" 指代的是动物。

\section{模型架构深度剖析}

本次实验复现的模型架构如图 \ref{fig:overall} 所示,这是一个经典的 Encoder-Decoder 结构。

\begin{figure}[htbp]
  \centering
  \framebox{\parbox{0.95\textwidth}{\centering
    \vspace{0.5cm}
    \includegraphics[width=0.95\textwidth]{images/overall_high_res.png}
    \vspace{0.5cm}
  }}
  \caption{\textbf{Transformer 整体架构图。} 
  (左) \textbf{编码器}:负责将源语言序列编码为高维语义矩阵。
  (右) \textbf{解码器}:利用自回归方式生成目标语言。
  (中) \textbf{交叉注意力}:解码器利用 Query 查询编码器的输出 (K, V)。}
  \label{fig:overall}
\end{figure}

\subsection{多头注意力 (Multi-Head Attention)}
单个注意力头可能只能捕捉一种类型的关系。为了让模型同时关注不同子空间的信息,Transformer 引入了多头注意力机制。具体做法是将 $Q, K, V$ 分别投影到 $h$ 个不同的子空间,在每个子空间独立计算注意力,最后将结果拼接起来:
\begin{equation}
\text{MultiHead}(Q, K, V) = \text{Concat}(\text{head}_1, \dots, \text{head}_h)W^O
\end{equation}
其中 $\text{head}_i = \text{Attention}(QW_i^Q, KW_i^K, VW_i^V)$。

直观地理解,不同的头可以学习关注不同类型的信息:有的头可能关注语法结构(主语在哪里),有的头可能关注语义指代("it"指的是什么),有的头可能关注位置关系(相邻的词)。这种设计大大增强了模型的表达能力。

\subsection{编码器 (Encoder):语义提取器}
编码器由 $N$ 层堆叠而成(本实验中 $N=3$,原论文中 $N=6$)。每一层包含两个核心子层:多头自注意力层和前馈网络层。前馈网络是一个简单的两层全连接网络,对每个位置独立应用:
\begin{equation}
\text{FFN}(x) = \max(0, xW_1 + b_1)W_2 + b_2
\end{equation}
这里使用 ReLU 激活函数提供非线性变换能力。

每个子层都配有\textbf{残差连接}和\textbf{层归一化},即 $\text{LayerNorm}(x + \text{Sublayer}(x))$。残差连接允许梯度直接流向底层,解决了深层网络的退化问题;层归一化则稳定了训练过程。

\subsection{解码器 (Decoder):自回归生成器}
解码器的结构与编码器类似,但有两个关键差异。

第一是 \textbf{Masked Self-Attention (掩码自注意力)}。在训练时,我们将整个目标句子一次性输入解码器,但不能让模型在预测第 $t$ 个词时看到第 $t+1$ 及之后的词(否则就是作弊了)。解决方案是使用一个下三角掩码矩阵,将未来位置的注意力分数设为 $-\infty$,经过 Softmax 后这些位置的权重就变成了 0。

第二是 \textbf{Cross Attention (交叉注意力)}。这是连接源语言和目标语言的桥梁。在这一层中,$Q$ 来自解码器上一层的输出,而 $K, V$ 来自编码器的最终输出。这使得解码器在生成每个词时,都能"查阅"源句子的相关部分,决定应该翻译哪个词。

\subsection{掩码机制可视化}
为了更清楚地理解掩码的作用,图 \ref{fig:mask} 展示了两种常用的掩码矩阵。

\begin{figure}[htbp]
  \centering
  \framebox{\parbox{0.9\textwidth}{\centering
    \vspace{0.5cm}
    \includegraphics[width=0.9\textwidth]{images/mask_high_res.png}
    \vspace{0.5cm}
  }}
  \caption{\textbf{掩码机制可视化。} 左图 (Padding Mask):展示了如何处理变长句子,矩阵会显式屏蔽掉 \texttt{<pad>} 对应的列(设为 $-\infty$),防止注意力机制关注到无意义的填充符。右图 (Causal/Look-ahead Mask):展示了下三角矩阵,这是解码器的核心,确保预测第 $t$ 个词时只能看到 $1$ 到 $t-1$ 的词,上三角部分(未来信息)被屏蔽。}
  \label{fig:mask}
\end{figure}

\subsection{位置编码 (Positional Encoding)}
由于 Self-Attention 机制本质上是集合运算,不具备顺序感。换句话说,如果我们打乱输入序列的顺序,Self-Attention 的输出也只是相应地打乱,不会有任何其他变化。这意味着"我爱你"和"你爱我"在纯 Attention 看来是一样的,这显然不行。

为了解决这个问题,Transformer 引入了位置编码。原论文使用正弦/余弦函数生成位置编码:
\begin{equation}
PE_{(pos, 2i)} = \sin(pos / 10000^{2i/d_{model}}), \quad PE_{(pos, 2i+1)} = \cos(pos / 10000^{2i/d_{model}})
\end{equation}
这种编码方式有一个有趣的性质:对于任意固定的偏移量 $k$,$PE_{pos+k}$ 可以表示为 $PE_{pos}$ 的线性函数,这使得模型能够学习到相对位置关系。位置编码与词嵌入相加后,作为 Transformer 的输入。

\section{实验设置与数据处理}

\subsection{数据集}
实验使用的是英中平行语料,包含训练集 75,134 条、验证集 1,748 条、测试集 2,001 条。数据已经过 BPE 分词处理,但这里有一个重要的细节需要说明。

\subsection{数据格式的"坑"}
拿到数据后,我最初以为 \texttt{@@} 是标准 BPE 中的续接标记(表示当前 token 与下一个 token 应该拼接)。但仔细观察后发现,几乎每个 token 后面都有 \texttt{@@},只有句末才没有。这说明在这份数据中,\texttt{@@} 实际上是\textbf{分词符号},类似于空格的作用。

这个发现很重要,因为如果按照标准 BPE 的方式处理,会导致分词完全错误。代码中需要特别处理这一点:对于从零训练的 Transformer,我们用 \texttt{@@} 作为分隔符切分 token;对于 mBART 预训练模型,则需要先将 \texttt{@@} 去掉还原成原始文本,再交给 mBART 自带的 tokenizer 处理。

\subsection{模型配置}
从零训练的 Transformer 采用了相对较小的配置:$d_{model}=256$,8 个注意力头,编码器和解码器各 3 层,前馈网络维度 512。这样的配置在 RTX 4090 上训练 10 个 epoch 大约需要几分钟。

\section{实验结果与分析}

\subsection{从零训练的 Transformer:一场"灾难"}

训练过程本身很顺利,loss 从 8.25 稳步下降到 6.88。然而,当我满怀期待地运行评估时,BLEU 分数只有 \textbf{0.03}。

更有趣的是翻译样例。我随机抽取了几条验证集数据,结果如表 \ref{tab:transformer_samples} 所示。

\begin{table}[htbp]
\centering
\caption{从零训练 Transformer 的翻译样例}
\label{tab:transformer_samples}
\begin{tabularx}{\textwidth}{lL}
\toprule
\textbf{类型} & \textbf{内容} \\
\midrule
源文 & new Questions Over California Water Project \\
参考 & 加利福尼亚州水务工程的新问题 \\
预测 & 我喜欢你说你 \emoji{grinning-face} \\
\midrule
源文 & the \$248 million in preliminary spending... \\
参考 & 这两条隧道的2.48亿美元初期费用支出... \\
预测 & 在1998年,19年1月年1月年1月... \\
\midrule
源文 & on Wednesday, state lawmakers ordered... \\
参考 & 一些州议员也于周三责令... \\
预测 & 在意大利的餐厅的餐厅和花园和花园... \\
\bottomrule
\end{tabularx}
\end{table}

看到这些结果,我忍不住笑出了声。模型似乎学会了几个"万能回答",无论你问什么,它都会自信地回复"我喜欢你说你"或者开始疯狂重复某个词组。这是典型的小模型训练不足的表现:模型还没学会真正的翻译映射,只是在随机拼凑训练数据中的高频词。

\subsection{交互模式的"名场面"}

为了进一步验证这个猜想,我进入了交互模式,手动输入一些句子测试。结果更加离谱:

\begin{lstlisting}[language=bash,caption=交互模式测试]
源句子> new@@ Questions@@ Over@@ California@@ Water@@ Project
译文: 请求的URL:/lll

源句子> hello
译文: 我喜欢你说你

源句子> 你好帅
译文: 我喜欢你说你

源句子> 红红火火恍恍惚惚
译文: 我喜欢你说你
\end{lstlisting}

无论输入什么——英文、中文、甚至乱码——模型都会坚定不移地输出"我喜欢你说你"。这大概是它在训练数据中学到的最"安全"的回答了 \emoji{grinning-face}

顺便说一下,交互模式需要输入 \texttt{@@} 分隔的英文格式(与训练数据一致),输入中文会全部变成 \texttt{<UNK>}。不过以这个模型的水平,输入格式正确与否似乎也没什么区别...

\subsection{问题分析}

从零训练效果差的原因其实很明显。首先是\textbf{模型太小},$d_{model}=256$、3 层的配置相比生产级翻译模型差了好几个数量级。其次是\textbf{训练不足},虽然 loss 在下降,但 10 个 epoch 远远不够让模型收敛到一个好的状态。最后是\textbf{数据量有限},7.5 万条平行语料对于从零训练一个翻译模型来说确实偏少。

不过话说回来,这个实验的目的本来就是理解 Transformer 的原理,而不是训练一个能用的翻译系统。从这个角度看,实验是成功的——我们确实从零实现了一个能跑的 Transformer,只是它还没学会翻译而已 \emoji{slightly-smiling-face}

\section{拓展:mBART-50 预训练模型}

既然从零训练效果不佳,那不如试试站在巨人的肩膀上。mBART-50 是 Facebook 发布的多语言预训练模型,支持 50 种语言之间的互译,参数量约 6 亿。

\subsection{一个小插曲:多语言的"惊喜"}

第一次运行 mBART 评估时,我被输出惊呆了:

\begin{lstlisting}[language=bash,caption=mBART 的多语言输出]
源文: new Questions Over California Water Project
参考: 加利福尼亚州水务工程的新问题
预测: neue Fragen uber das Wasserprojekt Kalifornien  (德语!)

源文: critics and a state law maker say...
预测: क्रिटिक और एक राज्य के विधायक कहते हैं...  (印地语!)

源文: critics said the government funding...
预测: Les critiques ont dit que le gouvernement...  (法语!)
\end{lstlisting}

模型把英文翻译成了德语、印地语、法语、西班牙语...就是不翻译成中文 \emoji{thinking-face}

原因是 mBART-50 是多语言模型,需要显式指定目标语言。我忘了设置 \texttt{forced\_bos\_token\_id} 参数,导致模型随机选择了输出语言。修复后,一切正常了。

\subsection{预训练模型的威力}

修复后的评估结果令人满意:验证集 Loss 2.90,BLEU \textbf{0.51}。这个 BLEU 分数是从零训练模型的 \textbf{17 倍}!

翻译样例也终于像样了:

\begin{table}[htbp]
\centering
\caption{mBART-50 预训练模型的翻译样例}
\label{tab:mbart_samples}
\begin{tabularx}{\textwidth}{lL}
\toprule
\textbf{类型} & \textbf{内容} \\
\midrule
源文 & new Questions Over California Water Project \\
参考 & 加利福尼亚州水务工程的新问题 \\
预测 & 加利福尼亚水项目新问题 \emoji{check-mark} \\
\bottomrule
\end{tabularx}
\end{table}

虽然不是完美翻译,但至少是正确的中文,而且语义基本准确。

\subsection{BPE 分词痕迹的影响}

不过预训练模型也不是完美的。由于数据中的 BPE 分词痕迹(空格)没有完全还原,导致一些有趣的翻译错误。比如 "fun ding"(funding 的 BPE 切分)被翻译成了"娱乐"(fun = 娱乐)\emoji{grinning-face},"cri tics"(critics)被翻译成了"克里克"(音译)。

这说明数据预处理的重要性——即使是强大的预训练模型,也会被脏数据带偏。

\section{总结与思考}

\subsection{实验心得}

本次实验让我对神经机器翻译有了更深的理解。从理论上,Transformer 通过自注意力机制解决了 RNN 的并行计算和长距离依赖问题;从实践上,我体会到了从零训练小模型的艰难,以及预训练大模型的强大。

实验中遇到的各种"翻车"场景("我喜欢你说你"、多语言输出、"fun ding = 娱乐")虽然搞笑,但都是很好的学习素材。它们让我意识到:机器翻译不仅仅是搭建一个模型那么简单,数据处理、参数设置、模型选择每一步都可能出问题。

由于数据规模很大,我没有在从零训练的 Transformer 上花太多时间调参和训练。直接用了课件中的基础配置,主要目的是验证实现的正确性。如果有更多时间和计算资源,可以尝试增加模型规模、训练轮数,或者使用更复杂的优化技巧(如学习率调度、正则化等),相信效果会有所提升。

最后,预训练模型以 17 倍的 BLEU 优势完胜从零训练的小模型,这充分说明了大规模预训练的价值。在实际应用中,除非有特殊需求,否则直接使用或微调预训练模型是更明智的选择。由于时间关系,我没有进行微调(Fine-tuning)。如果有时间和资源,微调预训练模型或许能带来更好的效果。

\subsection{已知局限性}

当前实现存在一些局限性。首先是\textbf{单向翻译},代码只支持英语到中文的翻译方向,虽然 mBART 本身支持双向,但需要修改代码才能实现。其次是\textbf{解码策略},从零训练的 Transformer 只实现了贪婪解码,没有实现 beam search,这可能影响翻译质量。最后是\textbf{数据格式依赖},代码针对 \texttt{@@} 分隔格式做了特殊处理,不兼容标准 BPE 格式。

\vspace{3em}

\begin{center}
\textit{实验完成时间:2025年12月9日}
\end{center}

\end{document}
